\problemname{Primtalssummor}

\begin{figure}
\centering
\begin{verbatim}
 163
 547
+829
----
1539
\end{verbatim}
\caption{De tre tresiffriga talen är alla primtal och siffrorna 1..9 används precis en gång i de tre talen}
\end{figure}

I figuren ser du en vanlig addition av tre tresiffriga tal. Det lite annorlunda med talen är att de alla är \emph{primtal}. Dessutom ingår var och en 
av de nio siffrorna 1..9 precis en gång i de tre talen.

Skriv ett program som tar reda på hur många sådana additioner som kan bildas. Programmet ska dessutom ta reda på det \emph{högsta} respektive \emph{lägsta}
värdet hos dessa summor.

Observera att alla additioner, som innehåller samma tre tal, betraktas som samma addition.

\section*{Indata}
Problemet har ingen indata.

\section*{Utdata}
Skriv ut tre heltal: antalet additioner, det lägsta värdet hos en summa, och det högsta värdet hos en summa.

